\documentclass{article}
\usepackage{fontspec}
\usepackage{graphicx}% Include figure files
\usepackage{bm}% bold math
\usepackage{amsmath}% bold math
\usepackage{indentfirst}
\usepackage{subfigure}
\usepackage[margin=1.5cm]{geometry}
\setmainfont{文泉驛正黑}
\XeTeXlinebreaklocale "zh"
\XeTeXlinebreakskip = 0pt plus 1pt
\newcommand{\figref}[1]{Figure \ref{Fig:#1}.}
\newcommand{\mm}{\operatorname{mm}}
\newcommand{\cm}{\operatorname{cm}}
\newcommand{\inch}{\operatorname{inch}}
\newcommand{\px}{\operatorname{px}}

\begin{document}
\title{ACA HW2}
\author{D01943032 林裕盛, NTU GIEE}
\date{\today}
\maketitle

\section{目的}

計算相機中的兩個像素點在真實世界中的距離

\section{使用工具}

我們使用以下這三個工具:
\begin{itemize}
\item 文公尺(即捲尺)
\item Canon EOS M 相機一台
\item 承上,$18-55 \mm$ 鏡頭一個
\end{itemize}

\section{結果}
首先我把文公尺垂直掛在冰箱上面,得到如 \figref{photo} $2592\times 1728$ 的相片。
\begin{figure}[ht]
\centering
\includegraphics[width=0.8\textwidth]{IMG_0076.JPG}
\caption{拍出來的相片}\label{Fig:photo}
\end{figure}
冰箱到相機的距離約 $231 \cm$,因為地上有地磚,所以應該是蠻準的垂直距離。

我們假設是針孔相機模型,並從 EXIF 取得以下資訊(見 \figref{exif}):
\begin{itemize}
\item Focal plane Y-resolution (per unit) (R): $2894.47 \px$
\item Focal plane Y-resolution unit (U): inch
\item Focal length (f): $47 \mm$
\end{itemize}

\begin{figure}[ht]
\centering
\includegraphics[width=0.4\textwidth]{exif.png}
\caption{用看圖軟體看 EXIF 資訊}\label{Fig:exif}
\end{figure}
其實相機的 X-resolution 跟 Y-resolution 不太一樣,不過差距只有 $0.2\%$,加上尺幾乎是垂直,所以就當作一樣了。

接著我們量測相機中 $80 \cm,~7 \cm$兩點的圖片座標,見 \figref{gimp},座標在左下角。得:兩點在圖上的距離為 \[\sqrt{(1407-1381)^2+(37-1711)^2} = 1674 \px\]

\begin{figure}[ht]
\centering
\subfigure[@$80\cm$]{\includegraphics[width=0.4\textwidth]{80.png}}\qquad
\subfigure[@$7\cm$]{\includegraphics[width=0.4\textwidth]{7.png}}
\caption{用看圖軟體看圖片座標}\label{Fig:gimp}
\end{figure}

接著我們要換算為在 CMOS 晶片上的距離:
\[1674 \px \times \dfrac{2.54 \cm}{2894.47 \px} = 1.469 \cm\]
最後使用國中學到的相似形:(在 CMOS 晶片上的距離:焦距)=(物體大小:物距),得到物體大小為
\[1.469 \cm * \dfrac{231 \cm}{47 \mm} = 72.2 \cm\]
理論值是 $73 \cm$,還蠻準的。
\end{document}